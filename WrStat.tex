\documentclass[a4paper]{article}

%
% Packages
%

%% styling for this document
\usepackage{tex/docstyle}

\usepackage{amsmath}
\usepackage{amssymb}
\usepackage{dsfont}

\usepackage{graphicx}

\usepackage{tabularx}
\usepackage{multirow}
\usepackage{multicol}
\usepackage{booktabs}
\usepackage{colortbl}
\usepackage{array}

\usepackage{enumitem}
\usepackage{parskip}

\usepackage{polyglossia}
\setmainlanguage[variant=swiss]{german}


%
% Metadata
%

\title{}
\author{
  Naoki Pross%\footnote{University of Applied Sciences of Eastern Switzerland}
}
\date{}

%
% Document
%


\renewcommand{\P}[1]{\mathrm{P}(#1)}
\newcommand{\given}{\,|\,}
\newcommand{\E}[1]{\mathrm{E}(#1)}
\newcommand{\Var}[1]{\mathrm{Var}(#1)}
\newcommand{\Cov}[1]{\mathrm{Cov}(#1)}
\DeclareMathOperator{\med}{\mathrm{med}}

\begin{document}

{\bfseries \huge \noindent
  Formelblatt --- Wahrscheinlichkeit und Statistik
}

\vspace{5mm}

\renewcommand{\arraystretch}{2}

\aboverulesep=0pt
\belowrulesep=0pt

\rowcolors{3}{lightgray!20}{white}

\noindent
\begin{tabularx}{\linewidth}{%
    | >{\columncolor{black}\cellcolor{black}}p{3mm} m{10.5cm} X |
  }
  \hline
  & \textbf{Produktregel} \newline \(k\) Positionen m\"ussen unabh\"angig von einadner markiert werden, wobai \(n_i\) verschiedene Markierungen zur Verf\"ugung stehen.
  & \(\displaystyle n_1 n_2\cdots n_k = \prod_{i=1}^k n_i \)
  \\

  & \textbf{Permutation} \newline Auf wie viele Arten lassen sich \(n\) verschiedene Objekte anordnen?
  & \(\displaystyle P_n = n(n-1)(n-2)\cdots1 = n! \)
  \\

  & \textbf{Kombination} \newline Auf wie vielen Arten kann man \(k\) aus \(n\) verschiedenen Objekte ausw\"ahlen?
  & \(\displaystyle C_n^k = {n \choose k} = \frac{n!}{k!(n-k)!}\)
  \\

  \multirow{-7}{*}{\centering
    \rotatebox[origin = c]{90}{
      \textcolor{white}{\bfseries Kombinatorik}
    }
  }
  & \textbf{Variation} \newline Auf wie viele Arten kann man \(k\) mal unter \(n\) verschiedenen Objekten ausw\"ahlen?
  & \(\displaystyle V_{n,k} = n^k \)
  \\
  \hline
\end{tabularx}

\vspace{3mm}

\noindent
\begin{tabularx}{\linewidth}{%
    | >{\cellcolor{black}}p{3mm} m{5cm} X |
  }

  \hline

  & \textbf{Ereignis}
  &
  \(
    \Omega = (\text{sicheres Ereignis}) \quad
    \emptyset = (\text{unm\"ogliches Ereignis}) \quad
    A,B \subseteq \Omega
  \)
  \newline
  \(
    A \cap B = (A \text{ und } B) \quad
    A \cup B = (A \text{ oder } B) \quad
    \bar{A} = \Omega \setminus A = (\text{nicht } A)
  \)
  \\[5pt]

  & \textbf{Wahrscheinlichkeit}
  & \(\displaystyle
    \mathrm{P}: \Omega \to [0;1] \quad
    \P{\emptyset} = 0 \quad
    \P{\Omega} = 1 \quad
    \P{\bar{A}} = 1 - \P{A}
  \) \newline
  \(A\) und \(B\) unabh\"angig \(\iff \P{A \cap B} = \P{A}\cdot\P{B}\)
  \\[5pt]

  & \textbf{Bedingte Wahrscheinlichkeit} \newline Wahrscheinlichkeit von \(A\) wenn \(B\) bereits eingetreten ist
  & \(\displaystyle
    \P{A \given B} = \frac{\P{A \cap B}}{\P{B}}
    \stackrel{\text{\footnotesize unabh.}}{=} \P{A} \qquad
    \P{\bar{A} \given B} = 1 - \P{A \given B}
  \)
  \\[5pt]

  & \textbf{Totale Wahrscheinlichkeit}
  & \(\displaystyle
    \P{A} = \sum_i \P{A \given B_i} \qquad
    A \subset \bigcup_i B_i
  \)
  \\[8pt]

  & \textbf{Satz von Bayes}
  & \(
  \P{A \given B} \cdot \P{B} = \P{B \given A} \cdot \P{A} = \P{A \cap B}
  \)
  \\[5pt]

  %% TODO: fix this dimexpr
  & \multicolumn{2}{p{\dimexpr\linewidth-12mm} | }{
    \textbf{Experimente} \newline
    In einem Laplace Experiment haben alle Elementarereignisse die gleiche Wahrscheinlichkeit. In einem Bernoulli Experiment es gibt nur 2 Ereignisse \(A\) und \(\bar{A}\) mit Wahrscheinlichkeiten \(p\) und \(1-p\).
  }
  \\[5pt]

  & \textbf{Zufallsvariable}
  & \(
    X : \Omega \to U \subseteq \mathbb{R} \quad
    \text{Ereignisse} \subset \Omega \,\text{ wie }\, \{ X = k \}, \{ X \leq x \}, \{ X > x \}
  \)
  \\[5pt]

  & \textbf{Erwartungswert}
  & \(\displaystyle
    \E{X} = \sum_{x \in U} x\cdot \P{X = x} \qquad
    \E{X + Y} = \E{X} + \E{Y}
  \)
  \newline
  \(
    \E{\lambda X} = \lambda \E{X} \qquad
    \E{XY} \stackrel{\text{\footnotesize unabh.}}{=} \E{X}\E{Y}
  \)
  \\[5pt]

  & \textbf{Varianz} \newline Mass f\"ur Streuung der Werte % \newline Quadratische Abweichung
  & \(\Var{X} = \E{(X - \E{X})^2} = \E{X^2} - \E{X}^2\) \newline
  \(
    \Var{\lambda X} = \lambda^2 \Var{X} \qquad
    \Var{X + Y} \stackrel{\text{\footnotesize unabh.}}{=} \Var{X} + \Var{Y} 
  \)
  \\[5pt]

  & \textbf{Covarianz}
  & \(\Cov{X,Y} = \E{XY} - \E{X}\E{Y} \stackrel{\text{\footnotesize unabh.}}{=} 0\)
  \\[5pt]

  \multirow{-21}{*}{\centering
    \rotatebox[origin = c]{90}{
      \textcolor{white}{\bfseries Wahrscheinlichkeit}
    }
  }
  & \textbf{Satz von Tschebyscheff}
  &
  \\[5pt]
  \hline
\end{tabularx}

\vspace{3mm}

\noindent
\begin{tabularx}{\linewidth}{%
    | >{\cellcolor{black}}p{3mm} m{4.5cm} X |
  }
  \hline
  & \textbf{Verteilungsfunktion}
  &
  ZV ist verteilt \(X \sim \mathcal{V}\) mit \(F : \mathbb{R} \to [0,1]\) monoton steigend \newline
  \(
    F(x) = \P{X \leq x} \qquad
    F(x\to\infty) = 1\) \qquad
    \(F(x\to-\infty) = 0
  \)
  \\[8pt]

  & \textbf{Median}
  & \(\displaystyle \med X = \inf \left\{ x : F(x) = 0.5 \right\} \)
  \\[4pt]

  & \textbf{Dichtefunktion}
  & \(\displaystyle
    \varphi(x) = \frac{dF}{dx} \qquad
    \P{a \leq X \leq b} = \int_a^b \varphi \,dx \qquad
    1 = \int_\mathbb{R} \varphi \,dx
  \)
  \\[8pt]

  & \textbf{Erwartungswert}
  & \(\displaystyle
    \E{X} = \int_\mathbb{R} x \varphi(x) \,dx \qquad
    \E{X^n} = \int_\mathbb{R} x^n \varphi(x) \, dx
  \)
  \\[8pt]

  & \textbf{Variablentransformation}
  & \(\displaystyle
    Y = g(X) \qquad
    \varphi_Y = \frac{\varphi_X}{g'} \circ g^{-1}
  \)
  \\[8pt]

  & \textbf{Standardisierung}
  & \(\displaystyle
    X \sim \mathcal{N}(\mu, \sigma) \qquad
    Z = \frac{X - \mu}{\sigma} \sim \mathcal{N}(0,1)
  \)
  \\[8pt]

  \multirow{-9}{*}{\centering
    \rotatebox[origin = c]{90}{
      \textcolor{white}{\bfseries W'keitsverteliung}
    }
  }
  & \textbf{Rechenregeln}
  &
  \\[8pt]
  \hline
\end{tabularx}

\vspace{3mm}

{
\noindent
\renewcommand{\arraystretch}{2.1}
\begin{tabularx}{\linewidth}{%
    | >{\cellcolor{black}}p{3mm} m{4.5cm} X c c c |
  }

  \hline
  & \bfseries Name
  & \(X \sim\)
  & \(\varphi(x) \text{ oder } \P{X = k}\)
  & \(\E{X}\)
  & \(\Var{X}\)
  \\[5pt]
  \hline
  
  & \textbf{Gleichverteilung} \newline Laplace Experimente
  & \(\displaystyle \mathcal{U}(a,b)\)
  & \(\displaystyle \frac{1}{b-a} \cdot \mathds{1}_{[a,b]} \)
  & \(\displaystyle \frac{a + b}{2}\)
  & \(\displaystyle \frac{(b - a)^2}{12} \)
  \\[5pt]

  & \textbf{Exponentialverteilung} \newline Halbwertszeit \(t_\frac{1}{2} = \log(2)/a\)
  & \(\displaystyle \mathcal{E}(a)\)
  & \(\displaystyle ae^{-ax} \cdot \mathds{1}_{[0,\infty)} \)
  & \(\displaystyle \frac{1}{a}\)
  & \(\displaystyle \frac{1}{a^2}\)
  \\[5pt]

  & \textbf{Normalverteilung} \newline Viele unabh. ZV
  & \(\displaystyle \mathcal{N}(\mu, \sigma)\)
  & \(\displaystyle \frac{1}{\sigma \sqrt{2\pi}} \cdot e^{-(x-\mu)^2 / 2\sigma^2} \)
  & \(\displaystyle \mu\)
  & \(\displaystyle \sigma^2 \)
  \\[5pt]

  & \textbf{Potenzverteilung} \newline Pareto Verteilung
  & \(\displaystyle \mathrm{Pow}(x_\textrm{m}, \alpha)\)
  &
  % \(\displaystyle \frac{\alpha - 1}{x_\textrm{m}} \left(
  %   \frac{x}{x_\textrm{m}}
  % \right)^{-\alpha} \mathds{1}_{[x_\textrm{m},\infty)}\)
  & \(\displaystyle x_\textrm{m} \cdot \frac{\alpha - 1}{\alpha - 2}\)
  &
  % \(\displaystyle \left(
  %   \frac{\alpha - 1}{\alpha - 3} - \left(
  %     \frac{\alpha - 1}{\alpha - 2}
  %   \right)^2
  % \right) x_\textrm{m}^2\)
  \\[5pt]

  & \textbf{Chi--Quadrat V.} \newline F\"ur das \(\chi^2\) Test
  & \(\displaystyle \mathcal{X}^2(k)\)
  &
  &
  &
  \\[5pt]

  \hline

  & \textbf{Geometrische V.}
  & \(\displaystyle \mathcal{G}(p)\)
  & \(\displaystyle p(1-p)^k \)
  & \(\displaystyle \frac{1}{p}\)
  & \(\displaystyle \frac{1-p}{p^2}\)
  \\[5pt]

  & \textbf{Hypergeometrische V.}
  & \(\displaystyle \mathcal{H}(N,R,n)\)
  & \({R \choose k}{N-R \choose n-k} / {N \choose n} \)
  & \(\displaystyle \frac{nR}{N}\)
  & \(\displaystyle \frac{nR}{N} \left(1 - \frac{R}{N}\right) \frac{N - n}{N - 1}\)
  \\[5pt]

  & \textbf{Poissonverteilung} \newline Seltener Ereignisse
  & \(\displaystyle \mathcal{P}(\lambda)\)
  & \(\displaystyle \frac{\lambda^k}{k!} e^{-\lambda} \)
  & \(\displaystyle \lambda\)
  & \(\displaystyle \lambda\)
  \\[5pt]

  \multirow{-12}{*}{\centering
    \rotatebox[origin = c]{90}{
      \textcolor{white}{\bfseries Katalog von W'keitsverteliungen}
    }
  }
  & \textbf{Binomialverteilung} \newline Bernoulli Experimente \newline
  & \(\displaystyle \mathcal{B}(n,p)\)
  & \(\displaystyle {n \choose k} p^k (1 - p)^{n - k} \)
  & \(\displaystyle np\)
  & \(\displaystyle np(1 - p) \)
  \\[5pt]

  %% TODO: fix this dimexpr and manual cell color
  &
  \multicolumn{5}{p{\dimexpr\linewidth-12mm} | }{
    % \cellcolor{lightgray!20}
    \cellcolor{white}
    F\"ur grosse \(n\) wird \(\mathcal{B}(n,p)\)
    \(\displaystyle \approx \mathcal{N}\left(\mu = np, \sigma = \sqrt{np(1-p)}\right) \)
    und f\"ur kleine \(p\) (selten) ist
    \(\displaystyle \approx \mathcal{P}\left(\lambda = np\right) \).
  }
  \\[4pt]

  \hline
\end{tabularx}
}

\vspace{3mm}

\noindent
\setlength{\parskip}{55pt}
\begin{tabularx}{\linewidth}{%
    | >{\cellcolor{black}}p{3mm} m{4.5cm} X |
  }
  \hline


  & \textbf{Regression} \newline Lineares Modell
  & \(\displaystyle 
    \text{ZV } X, Y \qquad
    y \approx ax + b \qquad
    a = \frac{\Cov{X,Y}}{\Var{X}} \qquad
    b = \E{Y} - a\E{X}
  \)
  \\
  & \textbf{Regressionskoeffizient}
  & \(\displaystyle
    r = \frac{\Cov{X,Y}}{\sqrt{\Var{X}\Var{Y}}} \qquad
    r^2 \approx 1 \implies \text{gute Approx.}
  \)
  \\

  &&\\
  &&\\
  &&\\
  &&\\

  \multirow{-5}{*}{\centering
    \rotatebox[origin = c]{90}{
      \textcolor{white}{\bfseries Sch\"atzen}
    }
  }
  & \textbf{}
  &
  \\
  \hline
\end{tabularx}

\noindent
\begin{tabularx}{\linewidth}{%
    | >{\cellcolor{black}}p{3mm} m{4.5cm} X |
  }
  \hline

  &&\\
  &&\\
  &&\\
  &&\\

  \multirow{-5}{*}{\centering
    \rotatebox[origin = c]{90}{
      \textcolor{white}{\bfseries Hypotesentesten}
    }
  }
  & \textbf{}
  &
  \\
  \hline
\end{tabularx}

\end{document}
